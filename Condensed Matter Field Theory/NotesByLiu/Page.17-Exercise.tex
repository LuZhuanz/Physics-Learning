\documentclass[12pt, a4paper, oneside]{ctexart}
\usepackage{amsmath, amsthm, amssymb, appendix, bm, graphicx, hyperref, mathrsfs,hep,color}
\begin{document}
\textcolor{blue}{Page.17-Exercise}\\
规范变换下${{A}_{\mu }}{{j}^{\mu }}\to \left( {{A}_{\mu }}+{{\partial }_{\mu }}\Gamma  \right){{j}^{\mu }}$
,然而
\[\int{{{d}^{4}}x\text{ }\left( {{\partial }_{\mu }}\Gamma  \right){{j}^{\mu }}}=\int{{{d}^{4}}x\text{ }{{\partial }_{\mu }}\left( \Gamma {{j}^{\mu }} \right)}-\int{{{d}^{4}}}x\text{ }\Gamma {{\partial }_{\mu }}{{j}^{\mu }}\]
由于场在边界取值为0,同时${{\partial }_{\mu }}{{j}^{\mu }}=0$,故上式为0.\\
但${{A}_{\mu }}{{A}^{\mu }}\to {{A}_{\mu }}{{A}^{\mu }}+{{A}_{\mu }}{{\partial }^{\mu }}\Gamma +{{\partial }_{\mu }}\Gamma {{A}^{\mu }}+{{\partial }_{\mu }}\Gamma {{\partial }^{\mu }}\Gamma $,对一般的$\Gamma$而言
\[\int{{{d}^{4}}x\text{ }\left( {{A}_{\mu }}{{\partial }^{\mu }}\Gamma +{{\partial }_{\mu }}\Gamma {{A}^{\mu }}+{{\partial }_{\mu }}\Gamma {{\partial }^{\mu }}\Gamma  \right)\ne 0}\]

\end{document}